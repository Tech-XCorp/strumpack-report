\documentclass{acmsmall}
\usepackage{siunitx}
\usepackage{booktabs}
\usepackage{url}
\usepackage[ruled]{algorithm2e}
\renewcommand{\algorithmcfname}{ALGORITHM}
\SetAlFnt{\small}
\SetAlCapFnt{\small}
\SetAlCapNameFnt{\small}
\SetAlCapHSkip{0pt}
\IncMargin{-\parindent}

\newcommand{\dmcomment}[1]{\textbf{#1}}


\begin{document}

\title{Replicated Computational Results (RCR) Report for
  \textit{A distributed-memory package for dense Hierarchically
    Semi-Separable matrix computations using randomization}}

\author{Dominic Meiser
\affil{Tech-X Corporation, 5621 Arapahoe Avenue, Boulder CO, 80303, USA.}
}

\newcommand{\strumpack}{STRUMPack}

\begin{abstract}
  In this report we replicate a subset of the performance results
  in the paper ``A distributed-memory package for dense
  Hierarchically Semi-Separable matrix computations using
  randomization''.
\end{abstract}

\maketitle 

\section{Introduction}

The article~\cite{rouet:strumpack} presents details of a newly
developed library, \strumpack, for the construction of
hierarchically semi-separable matrices and for arithmetic with
these matrices.  Among other things,  \strumpack{} can convert
ordinary dense matrices into a hierarchical matrix.  Compression
is achieved by means of a low rank approximation for sub blocks
of the matrix.  The resulting hierarchical matrix can then be
factored efficiently and the factorization can be used to solve
linear systems with the original matrix.  All of this
functionality is available in a distributed memory architecture
using MPI.

In this report we replicate some of the computational results
presented in~\cite{rouet:strumpack}.  We focus compression,
factorization, and solve performance reported in Table III
of~\cite{rouet:strumpack}.  We also check the strong scaling
results reported in~Fig.~10 in~\cite{rouet:strumpack}.

Specifically, we carry out the following steps:
\begin{enumerate}
  \item Download \strumpack{} distribution package.
  \item Build \strumpack{} on a laptop.
  \item Build \strumpack{} on Edison~\cite{Edison} at NERSC.
  \item Run \strumpack{} on Edison to replicate performance
    measurements for QChem Toeplitz matrix and to check strong
    scaling for a dense matrix from boundary element formulation
    of an electromagnetic scattering problem..
\end{enumerate}

The following sections provide details on these steps.


\section{Building \strumpack{} on a laptop}

To verify that we could build the package we first get it to run
on a laptop.  This allows us to iron out any issues ahead of time
in an environment that we have complete control over before
attempting to do the same on an expensive computer like Edison.
Additionally, this provides confidence that \strumpack{} can be
used on generic computers and not just in the specific
environment used in~\cite{rouet:strumpack}.

We obtain the sources from the \strumpack{} distribution website,
\begin{verbatim}
STRUMPACK_URL=http://portal.nersc.gov/project/sparse/strumpack \
  TARBALL_NAME=STRUMPACK-Dense-1.1.1.tar.gz \
  wget $STRUMPACK_URL/$TARBALL_NAME
tar xf STRUMPACK-Dense-1.1.1.tar.gz
\end{verbatim}

The source distribution contains a 20 page manual with
introduction, installation instructions, background on the
algorithms, and an API reference.  A readme file
provides concise instructions on how to build \strumpack{}.

\strumpack{}'s third party library requirements are rather
minimal: MPI, BLAS/LAPACK, and ScaLAPACK.  For our laptop builds
We satisfy these requirements as follows:
\begin{itemize}
\item MPI:\@ mpich  3.1.4
\item BLAS/LAPACK:\@ binary distribution on centos 6
\item ScaLAPACK:\@ Reference implementation from netlib~\cite{netlib}
\end{itemize}

All code is compiled with gcc version 4.9.3.

\strumpack{}'s build system consists of makefiles.  They are
customized by writing a Makefile.inc that defines system specific
variables.  Template Makefile.inc files are provided for gnu
based linux systems, edison, and hopper.  To build on our laptop
we modify Makefile.gnu to point to our mpi, blas, lapack, and
scalapack installation.  With these modifications we successfully
build the C++, C, and Fortran examples.  We don't verify building
of the matlab bindings because our laptop does not have matlab
installed.  We are able to execute all example binaries and the
output from them looks plausible.


\section{Building on Edison}

The majority of the performance results reported
in~\cite{rouet:strumpack} were obtained on Edison~\cite{Edison} at
NERSC.\@ To build \strumpack{} on Edison we use the default intel
programming environment, \verb!PrgEnv-intel/5.2.56!  with version
15.0.1 20141023 of the intel compilers, with module
\verb!cray-mpich/7.3.1! for MPI support.  The template
\verb!Makefile.edison! provided with the source distribution of
STRUMPACK works without modifications with these modules.  This
compiles all sources with \verb!-O3!.


\section{Performance on Edison}

We use the example \verb!solve.cpp! to reproduce the results for
the QChem Toeplitz matrices in Table~III
in~\cite{rouet:strumpack}.  Currently there is no mechanism to
control the example program via command line options.  Thus we
modify the \verb!solve.cpp! source code and rebuild for every set
of parameters.  We set the problem size to \verb!n=80,000! and
change the number of right hand sides to \verb!nrhs=1!.  Details
of the computation are controlled via options of the
computational object \verb!sdp! of type
\verb!StrumpackDensePackage!.  After discussion with the authors
of~\cite{rouet:strumpack} we configure \verb!sdp! as follows to
match the settings used in~\cite{rouet:strumpack}:
\begin{verbatim}
sdp.use_HSS=true;
sdp.tol_HSS=1e-8;
sdp.levels_HSS=13;
sdp.min_rand_HSS=250;
sdp.lim_rand_HSS=5;
sdp.inc_rand_HSS=10;
sdp.max_rand_HSS=250;
sdp.steps_IR=10;
sdp.tol_IR=1e-8;
\end{verbatim}
The setting \verb!sdp.use_HSS=true! specifies that HSS
compression be used.  By setting this parameter to false it is
possible to test the solution of the system via ScaLAPACK instead
of \strumpack{}.  The parameter \verb!sdp.tol_HSS! sets
$\epsilon$ in Table III in~\cite{rouet:strumpack}.
\verb!sdp.levels_HSS! specifies the maximum number of levels in
the HSS tree.  The parameters \verb!sdp.min_rand_HSS!,
\verb!sdp.lim_rand_HSS!, \verb!sdp.inc_rand_HSS!, and
\verb!sdp.max_rand_HSS! configure the adaptive sampling of random
vectors used for the construction of the HSS matrix in
\strumpack{}.  By setting minimum and maximum numbers of random
vectors equal to one another,
\verb!sdp.min_rand_HSS==sdp.max_rand_HSS!, we effectively disable
the adaptive sampling algorithm and use a fixed number of
vectors, 250 in the example above.  The number of sampling
vectors is adjusted for the different tolerances $\epsilon$ since
the maximum rank depends on the precision of the reduced rank
representation.  The parameter \verb!sdp.steps_IR! determines the
maximum number of iterations used for iterative refinement and
\verb!sdp.tol_IR!  specifies the precision to which the linear
system is being solved.  Reporting of performance numbers relies
on the \verb!sdp.print_statistics()! function.

Initially we run with verification of compression enabled.  This
is a very memory intensive operation and thus requires more
compute nodes than were used in~\cite{rouet:strumpack}.  After
verifying the accuracy of the compression we disable this
verification feature and reduce the number of nodes to the
configuration used in~\cite{rouet:strumpack}.  Specifically, we
use four compute nodes with 16 MPI processes on each of them.

Our performance results are shown in
table~\ref{table:QChemToeplitz} along with the results reported
in~\cite{rouet:strumpack}.  The timings we obtain for the
compression step are very close to the results reported
in~\cite{rouet:strumpack}.  The timings for factorization and
solution vary a bit.  But because these stages are only fractions
of a second long they are much more susceptible to fluctuations.
We find that they are close enough to consider the results
of~\cite{rouet:strumpack} replicated.
\begin{table}
  \caption{Summary of performance measurements.  All numbers
    indicate times in seconds.  The timings
    from~\cite{rouet:strumpack} are indicated in parenthesis.}
\label{table:QChemToeplitz}
  \begin{tabular}{ccccc}
\toprule
$\epsilon$ & Max.\@ rank & Compression & Factorization & Solution + IR\\
\midrule
    \SI{1.0e-8} & 171 (169) & 19.0 (19.0) & 0.04 (0.04) & 0.2 (1.5)\\
    \SI{1.0e-6} & 147 (147) & 18.2 (17.6) & 0.07 (0.03) & 0.7 (2.0)\\
    \SI{1.0e-4} & 121 (120) & 16.9 (16.7) & 0.04 (0.02) & 0.4 (5.6)\\
\bottomrule
  \end{tabular}
\end{table}
\dmcomment{Sherry, Francois, I think I'm getting very different
  times for the solution step because I'm solving to a smaller
  accuracy (sdp.tol\_IR in the code).  Could you confirm that,
  please?  What should I use for sdp.tol\_IR?}

Besides the timing results we also made sure that the maximum
rank reported by strumpack agrees with~\cite{rouet:strumpack}.
We find that in our runs the maximum rank is within 1\% of the
ranks reported in~\cite{rouet:strumpack}.  We have not been able
to determine the cause of the small discrepancies between the
ranks reported in our runs and~\cite{rouet:strumpack} in the case
of $\epsilon=$\SI{1.0e-8}.  But the two are certainly very close
to one another.


\section{Replication of strong scaling results}

To replicate the strong scaling performance reported in Fig.~10
in~\cite{rouet:strumpack} we use the same example program,
\verb!driver.cpp! provided by the authors
of~\cite{rouet:strumpack}.  This executable is not part of the
release distribution of~\cite{rouet:strumpack}.  We build it on Edison
using
\begin{verbatim}
CC driver.cpp -I ${STRUMPACK_SRC_DIR}
\end{verbatim}
The directory \verb!STRUMPACK_SRC_DIR! is the directory in which
the \strumpack{} sources are locate.  The driver program loads
the matrices from Francois' scratch directory on Edison.  The
matrices are stored in a proprietary binary format.


\section{Conclusion}

We replicate the results reported in~\cite{rouet:strumpack}.  The
STRUMPACK library is easy to build and contains sufficient
documentation for others to use the library.  We build the
library on a laptop and on Edison.  We reproduce the timing
results reported for a synthetic matrix as well as strong scaling
results for a matrix originating from an electromagnetic boundary
element problem.


\section{Acknowledgements}

We would like to thank Francois-Henry Rouet and Xiaoye S. Li for
valuable feedback and discussion and M. Heroux for guidance.
This research used resources of the National Energy Research
Scientific Computing Center, which is supported by the Office of
Science of the U. S. Department of Energy under Contract No. ???
\dmcomment{Hi Sherry, what contract should I acknowledge for
  mp127 time? Thanks.}

\bibliographystyle{abbrv}
\bibliography{strumpack}

\end{document}
