\documentclass{acmsmall}
\usepackage{url}
\usepackage[ruled]{algorithm2e}
\renewcommand{\algorithmcfname}{ALGORITHM}
\SetAlFnt{\small}
\SetAlCapFnt{\small}
\SetAlCapNameFnt{\small}
\SetAlCapHSkip{0pt}
\IncMargin{-\parindent}

\begin{document}

\title{Strumpack}
\author{Dominic Meiser
\affil{Tech-X Corporation, Boulder, CO 80303, USA.}
}


\begin{abstract}
Multifrequency media access control has been well understood in
general wireless ad hoc networks, while in wireless sensor networks,
researchers still focus on single frequency solutions. In wireless
sensor networks, each device is typically equipped with a single
radio transceiver and applications adopt much smaller packet sizes
compared to those in general wireless ad hoc networks. Hence, the
multifrequency MAC protocols proposed for general wireless ad hoc
networks are not suitable for wireless sensor network applications,
which we further demonstrate through our simulation experiments. In
this article, we propose MMSN, which takes advantage of
multifrequency availability while, at the same time, takes into
consideration the restrictions of wireless sensor networks. Through
extensive experiments, MMSN exhibits the prominent ability to utilize
parallel transmissions among neighboring nodes. When multiple physical
frequencies are available, it also achieves increased energy
efficiency, demonstrating the ability to work against radio
interference and the tolerance to a wide range of measured time
synchronization errors.
\end{abstract}

\maketitle 

\begin{itemize}
  \item Download strumpack
  \item readme
  \item build on laptop
  \item build on edison
  \item run examples
  \item reproduce results

\end{itemize}

\section{First impressions}

Downloaded the source package from
\url{http://portal.nersc.gov/project/sparse/strumpack/STRUMPACK-Dense-1.1.1.tar.gz}

Unpacked using \verb!tar xf STRUMPACK-Dense-1.1.1.tar.gz!

In the source distribution there is a 20 page manual with
introduction, installation instructions, background on the
algorithms, and an API reference.  A 167 line long readme file
provides concise instructions on getting started.

Primary requirements:
- MPI
- blas/lapack
- scalapack

We use mpich v 3.1.4, along with reference lapack and blas.

scalapack 2.0.2

gcc 4.9.3


\section{building}

Modify Makefile.gnu:
\begin{itemize}
\item Set blas, lapack, and scalapack libraries
\item set compiler wrappers
\end{itemize}

Use that Makefile as an include in Makefile in examples.

Then make -j8


Compiles with a relatively large number of warnings.

Built C++, C, and Fortran example.  Didn't try matlab/octave example.

Ran a few examples.



\end{document}
